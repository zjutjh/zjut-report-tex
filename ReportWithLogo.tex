\documentclass[a4paper,12pt]{ctexart}
\usepackage{graphicx} %插入图片的宏包
\usepackage{float} %设置图片浮动位置的宏包
\usepackage{subfigure} %插入多图时用子图显示的宏包
\usepackage{fancyhdr} %设置页眉页脚的宏包
\usepackage[]{caption2} %设置图和表的格式的宏包
\usepackage{multirow} %合并多行单元格的宏包
\usepackage{longtable} %不宽但很长的表格可以用longtable宏包来进行分页显示
\usepackage{array} %一般用于数学公式中对数组或矩阵的排版
\usepackage{makecell}% makecell命令对表格单元格中的数据进行一些变换的控制。我们可以使用 \ 命令进行换行,也可以使用p{(宽度)}选项控制列表的宽度
\usepackage{threeparttable} %制作三线表格
\usepackage{booktabs}%s三线表格中的上中下直线线型设置宏包,在这个包中水平线被教程\toprule、midrule、buttomrule。
\usepackage{enumerate} %列举宏包

%页眉页脚设置
\pagenumbering{arabic}
\pagestyle{fancy}
\setlength{\headheight}{14pt}
\fancyhead[L]{}
\fancyhead[R]{}
\fancyhead[C]{\small{报告名称}}
\fancyfoot[C]{\thepage}

%标题序号长度设置
\setcounter{secnumdepth}{3}

%图片排版设置
\renewcommand{\figurename}{图} %重定义编号前缀词
\renewcommand{\captionlabeldelim}{.~} %重定义分隔符
 %\roman是罗马数字编号,\alph是默认的字母编号,\arabic是阿拉伯数字编号,可按需替换下一行的相应位置
\renewcommand{\thesubfigure}{(\roman{subfigure})}%此外,还可设置图编号显示格式,加括号或者不加括号
\makeatletter \renewcommand{\@thesubfigure}{\thesubfigure \space}%子图编号与名称的间隔设置
\renewcommand{\p@subfigure}{} \makeatother

%表头文字格式的详细设置
\renewcommand\theadset{\renewcommand\arraystretch{0.85}%
\setlength\extrarowheight{0pt}}%行距
\renewcommand\theadfont{\small}%字体
\renewcommand\theadalign{rt}%行列对齐
\renewcommand\theadgape{\Gape[0.5cm][2mm]}%上下垂直距离


\title{
  \begin{figure}[H]
    \centering
    \includegraphics[width=0.7\textwidth]{./img/校名.png}
  \end{figure} 
  \huge \textbf{报告名称} \\ 
  \large \textbf{(年级)}
  \begin{figure}[H]
    \centering
    \includegraphics[width=0.3\textwidth]{./img/校徽.png}
  \end{figure}
  \large 实验题目\hspace{0.7cm}\underline{\makebox[5.5cm]{实验1}} \\
  \large 学生姓名\hspace{0.7cm}\underline{\makebox[5.5cm]{派大星}} \\
  \large 专业班级\hspace{0.7cm}\underline{\makebox[5.5cm]{软件工程 2103}} \\
  \large 所在学院\hspace{0.7cm}\underline{\makebox[5.5cm]{计算机科学与技术学院}} \\
  \large 提交日期\hspace{0.7cm}\underline{\makebox[5.5cm]{2023年3月9日}} \\
}
\author{}
\date{}
\begin{document}
\maketitle
\newpage

\pagenumbering{roman}
\tableofcontents
\newpage

\listoffigures
\listoftables
\newpage

\pagenumbering{arabic}
\section{大标题一}

\subsection{中标题一}

\subsubsection{小标题一}

正文

\section{大标题二}

\subsection{中标题二}

\subsubsection{小标题二}

%单图
\begin{figure}[H] %H为当前位置,!htb为忽略美学标准,htbp为浮动图形
  \centering %图片居中
  \includegraphics[width=0.7\textwidth]{./img/校徽.png} %插入图片,[]中设置图片大小,{}中是图片文件名
  \caption{Main name 2} %最终文档中希望显示的图片标题
  \label{Fig.main2} %用于文内引用的标签
\end{figure}

%一个板块下两个字图
\begin{figure}[H]
  \centering  %图片全局居中
  \subfigure[name1]{
    \label{Fig.sub.1}
    \includegraphics[width=0.45\textwidth]{./img/校徽.png}}
  \subfigure[name2]{
    \label{Fig.sub.2}
    \includegraphics[width=0.45\textwidth]{./img/校名.png}}
  \caption{Main name}
  \label{Fig.main}
\end{figure}

%图片引用
图 \ref{Fig.main} 有两个字图,图 \ref{Fig.sub.1} 是浙工大校徽, 图 \ref{Fig.sub.2} 是浙工大校名.

%两个并排的独立图片
\begin{figure}[H]
  \centering %图片全局居中
  %并排几个图,就要写几个minipage
  \begin{minipage}[b]{0.45\textwidth} %所有minipage宽度之和要小于1,否则会自动变成竖排
    \centering %图片局部居中
    \includegraphics[width=0.8\textwidth]{./img/校徽.png} %此时的图片宽度比例是相对于这个minipage的,不是全局
    \caption{name 1}
    \label{Fig.1}
  \end{minipage}
  \begin{minipage}[b]{0.45\textwidth} %所有minipage宽度之和要小于1,否则会自动变成竖排
    \centering %图片局部居中
    \includegraphics[width=0.8\textwidth]{./img/校名.png}%此时的图片宽度比例是相对于这个minipage的,不是全局
    \caption{name 2}
    \label{Fig.2}
  \end{minipage}
\end{figure}

\begin{table}[htbp] %表格的浮动环境
  \centering\small
  \begin{threeparttable}
    \caption{表格样例}
    \begin{tabular}{lccc} %表格环境,{}中是单元格对齐方式,l左对齐,c居中,r右对齐
      \toprule %表头直线
      \makecell[c]{双行                                                            \\ 样例}    &   单列 & \multicolumn{2}{c}{组合列} \\
      \midrule %表中直线
      $\ln(y/pop)$                    & \makecell[c]{双行                          \\ 样例}  &    287.25* &     566.65 \\
      \midrule %表中直线
      \multirow{2}{*}{$\ln(y/pop)^2$} & $-$22.85*         & $-$16.58* & $-$35.57** \\ \cline{2-4}
                                      & $-$.29**          & $-$.31*   & $-$.37     \\
      \midrule %表中直线
      $Polity$                        & $-$3.20*          & $-$6.58*  & $-$6.70**  \\
      \midrule %表中直线
      $\ln(LandArea/pop)$             & $-$5.94           & $-$2.92*  & $-$13.02*  \\
      \midrule %表中直线
      Obs.                            & 36                & 41        & 38         \\
      \midrule %表中直线
      $R^2$                           & 0.16              & 0.68      & 0.62       \\
      \bottomrule %表底直线
    \end{tabular}
    \small
    备注: 可以在这里表格的备注
    \begin{tablenotes}
      \item[*] 在 5\% 水平上显着
      \item[**] 在 10\% 水平上显着
    \end{tablenotes}
  \end{threeparttable}
\end{table}
\newpage

% 无编号,但公式符号可以对齐的做法
\begin{enumerate}[align=right]
  \setlength{\leftmargin}{2em} %左边界
  \setlength{\parsep}{0ex} %段落间距
  \setlength{\topsep}{0ex} %列表到上下文的垂直距离
  \setlength{\itemsep}{0ex} %条目间距
  \setlength{\labelsep}{1em} %标号和列表项之间的距离,默认0.5em
  \setlength{\itemindent}{0em} %标签缩进量
  \setlength{\listparindent}{0em} %段落缩进量
  \item [ $c_1$] 行驶(汽车)的恒定行程时间,单位:分钟/次;
  \item [$c_2$] 乘坐捷运的常数行程时间,单位:分钟/次;
  \item [$\tau_1$] 一个旅行者开车的单位固定成本,单位:元/(人*次);
  \item [$\tau_2$] 旅客单程捷运票价,单位:元/(人*次);
  \item [$\varsigma_1$] 交通管理者公路单位运营成本,单位:元/(人*次);
\end{enumerate}

%无编号的一般做法
\begin{itemize}
  \item 春天
  \item 夏天
  \item 秋天
  \item 冬天
\end{itemize}

%有编号的一般做法
\begin{enumerate}[(1)]
  \item 白天
  \item 黑夜
\end{enumerate}

\end{document}
